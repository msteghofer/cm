\clearemptydoublepage
\chapter{Solvers}
\label{cha:solvers}

\section{Linear Solvers}

\subsection{Linear Direct Solvers}

\subsection{Linear Iterative Solvers}

\section{Nonlinear Solvers}

\subsection{Newton Solvers}

\subsubsection{Newton Line Search Solvers}

\subsubsection{Newton Trust Region Solvers}

\subsection{Quasi-Newton Solvers}

\subsubsection{Broyden-Fletcher-Goldfarb-Shanno (BFGS) Solvers}

\subsection{Sequential Quadratic Program (SQP) Solvers}

\section{Dynamic Solvers}

\section{Differential-Algebraic Equation (DAE) Solvers }

\section{Eigenproblem Solvers}

\section{Optimisation Solvers}

\section{Fast Marching Solver}

\noindent Currently OpenCMISS uses Fast Marching Methods (FMM) for solving Hamilton-Jacobi Equations. FMM was introduced by Sethian to Hamilton-Jacobi equations of static type by a finite difference discretization up-wind scheme \cite{sethian:1996}. This method allows speeding up the convergence of the classical iterative finite difference scheme. It computes the approximate solution in a finite number of steps and the complexity behaves as $O(N\ln N)$ where $N$ is the total number of nodes \cite{mauch:2003}. This method stands on two steps including selection of the proper wavefront point to continue marching and then calculation and updating unknown variable for its surroundings (See Figure \ref{fig:FMM}).

\begin{figure}[h!]
  \centering
    \includegraphics[width=0.75\textwidth]{../latex/epsfiles/HJE/FigFMM.eps}
  \caption{Schematic of FMM.}
  \label{fig:FMM}
\end{figure}

