\clearemptydoublepage
\chapter{Analytic Solutions}
\label{cha:analyticsolutions}


\section{Classical Field Class}
\clearpage
%\input{AnalyticSolutions/ClassicalFieldClass/GeneralisedLaplaceEquation.tex}
\clearpage
%\input{AnalyticSolutions/ClassicalFieldClass/PoissonEquation.tex}
\clearpage
%\input{AnalyticSolutions/ClassicalFieldClass/HelmholtzEquation.tex}
\clearpage
%\input{AnalyticSolutions/ClassicalFieldClass/WaveEquation.tex}
\clearpage
\input{AnalyticSolutions/ClassicalFieldClass/DiffusionEquation.tex}
\clearpage
%\input{AnalyticSolutions/ClassicalFieldClass/AdvectionDiffusionEquation.tex}
\clearpage
%\input{AnalyticSolutions/ClassicalFieldClass/ReactionDiffusionEquation.tex}
\clearpage
%\input{AnalyticSolutions/ClassicalFieldClass/BiharmonicEquation.tex}


\section{Elasticity Class}
\clearpage
%\input{AnalyticSolutions/ElasticityClass/LinearElasticity.tex}
\clearpage
%\input{AnalyticSolutions/ElasticityClass/FiniteElasticity.tex}
\clearpage

\section{Fluid Mechanics Class}
\clearpage
%\input{AnalyticSolutions/FluidMechanicsClass/StokesEquation.tex}
\clearpage
%\input{AnalyticSolutions/FluidMechanicsClass/DarcyEquation.tex}
\clearpage
%\input{AnalyticSolutions/FluidMechanicsClass/NavierStokesEquation.tex}
\clearpage
%\input{AnalyticSolutions/FluidMechanicsClass/PressurePoissonEquation.tex}
\clearpage
\input{AnalyticSolutions/FluidMechanicsClass/BurgersEquation.tex}

\section{Electromechanics Class}
\clearpage
%\input{AnalyticSolutions/ElectroMechanicsClass/ElectrostaticEquations.tex}
\clearpage
%\input{AnalyticSolutions/ElectroMechanicsClass/MagnetostaticEquations.tex}
\clearpage
%\input{AnalyticSolutions/ElectroMechanicsClass/MaxwellEquations.tex}



\section{Bioelectrics Class}
\clearpage
%\input{AnalyticSolutions/BioElectricsClass/BidomainEquation.tex}
\clearpage
%\input{AnalyticSolutions/BioElectricsClass/MonodomainEquation2.tex}
%\input{AnalyticSolutions/BioElectricsClass/MonodomainEquation.tex}

\section{Modal Class}
\clearpage

\section{Fitting Class}
\clearpage
%\input{AnalyticSolutions/Fitting/Fitting.tex}

\subsection{Hamilton-Jacobi Equation}

\subsubsection{Anisotropic Eikonal Solver}

Available analyticsolutions in the Hamilton-Jacobi Equations of Anisotropic Eikonal solver are as below:


\textbf{Analytic Solution 1:} Solving with $a_{ij}=\delta_{ij}$ and function $f(x,y)=1$. The computational domain is $\Omega=[-1,1]^2$, and $\partial\Omega$ is a source point at the origin with coordinates $(0,0)$. The exact solution is $$\phi(x,y)=\sqrt{x^2+y^2}$$

\begin{figure}[h]
  \centering
    \includegraphics[width=0.75\textwidth]{../latex/epsfiles/HJE/FigEx1Res.eps}
  \caption{Analytic Solution 1 results.}
  \label{fig:Ex1}
\end{figure}


\textbf{Analytic Solution 2:} Solving with $a_{ij}=\delta_{ij}$ and $$f(x,y)=2\pi \sqrt{10[\cos{(2\pi x)}\sin{(2\pi y)}]^2+[\sin{(2\pi x)}\cos{(2\pi y)}]^2}$$ The boundary is given by $\partial\Omega=\{(\frac{1}{4},\frac{1}{4}),(\frac{3}{4},\frac{3}{4}),(\frac{3}{4},\frac{1}{4}),(\frac{1}{4},\frac{3}{4}),(\frac{1}{2},\frac{1}{2})\}=\{1,1,-1,-1,0\}$, which consists of five isolated points. The computational domain is $\Omega=[0,1]^2$. $\phi(x,y)=0$ is prescribed at the boundary of the unit square. The exact solution for this case is $$\phi(x,y)=\sin{(2\pi x)} \sin{(2\pi y)}$$

\begin{figure}[h]
  \centering
    \includegraphics[width=0.75\textwidth]{../latex/epsfiles/HJE/FigEx2Res.eps}
  \caption{Analytic Solution 2 results.}
  \label{fig:Ex2}
\end{figure}


\textbf{Analytic Solution 3:} Solving with $a_{ij}=\delta_{ij}$ and $$f(x,y,z)=\frac{\pi}{2} \sqrt{\sin^2{(\pi + \frac{\pi}{2}x)}+\sin^2{(\pi + \frac{\pi}{2}y)}+\sin^2{(\pi + \frac{\pi}{2}z)}}$$ The computational domain is $\Omega=[-1,1]$ in $x$ and $y$ direction and $\Omega=[-1,1]^3$ at $z$ direction. The inflow boundary $\partial\Omega$ is a source at the origin $(0,0,0)$. The exact solution for this problem is $$\phi(x,y,z)=\cos{(\pi + \frac{\pi}{2}x})+\cos{(\pi + \frac{\pi}{2}y})+\cos{(\pi + \frac{\pi}{2}z})$$

\begin{figure}[h]
  \centering
    \includegraphics[width=0.75\textwidth]{../latex/epsfiles/HJE/FigEx3Res.eps}
  \caption{Analytic Solution 3 results.}
  \label{fig:Ex3}
\end{figure}


\textbf{Two Element Example:} Solving with anisotropic case for two element domain with constant function $f(x,y)=1$ along fiber directions and $f(x,y)=10$ in plane of isotropy.

\begin{figure}[h]
  \centering
    \includegraphics[width=0.75\textwidth]{../latex/epsfiles/HJE/FigEx4Res.eps}
  \caption{Two Element Example results.}
  \label{fig:Ex4}
\end{figure}


\subsubsection{Geodesic Solver}

\noindent The analyticsolutions for Geodesic solver which is included in the library is:

\textbf{GeodesicEx1} Solving within a cubic mesh domain, the aim is to find the geodesic path between one corner of the cube and one point on a face of the cube using this algorithm.
Depending on the application the Geodesic nodes list and Geodesic path lengths is available.

\begin{figure}[h]
  \centering
    \includegraphics[width=0.4\textwidth]{../latex/epsfiles/HJE/FigGeodesicExRes.eps}
  \caption{Geodesic path from the example.}
  \label{fig:GeodesicEx}
\end{figure}

\section{Optimisation Class}
\clearpage
