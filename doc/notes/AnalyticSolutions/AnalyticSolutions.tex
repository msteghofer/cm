\clearemptydoublepage
\chapter{Analytic Solutions}
\label{cha:analyticsolutions}


\section{Classical Field Class}
\clearpage
%\subsection{Generalised Laplace Equation}

Using the example of the generalised Laplace equation,
we illustrate here the general workflow from the governing equations
to the ``algorithmic'' form of the equations that is amenable
for implementation into OpenCMISS.
We thus start with the governing equations and subsequently introduce
the weak (variational) form of these equations.
We then shall switch to tensor notation which will be more amenable
to code implementation.
After breaking up the domain integrals arising in the formulation
into a sum of element integrals and introducing numerical integration,
we finally arrive at the ``algorithmic'' form of the equations
that is ready for implementation in the code.


\subsubsection{Governing equations:}
The generalised Laplace's equation on a domain $\Omega$ with boundary $\Gamma$
can be stated as
\begin{equation}
  \diverg{\pbrac{\fnof{\tensor{\sigma}}{\vect{x}}\gradient{\fnof{\phi}{\vect{x}}}}}=0  \quad \mathrm{on} \, \Omega \,,
  \label{eqn:Laplacesequation}
\end{equation}
where $\fnof{\phi}{\vect{x}}$ is the potential and $\fnof{\tensor{\sigma}}{\vect{x}}$ is
the (positive definite) conductivity tensor throughout the domain. Note that 
$\tensor{\sigma}=\tensor{I}$ gives the standard form of Laplace's equation \ie $\laplacian{\phi}=0$.

To complete the description of the boundary value problem,
the governing equation~(\ref{eqn:Laplacesequation}) is supplemented
by suitable boundary conditions on the domain boundary~$\Gamma$.
These boundary conditions can either be of Dirichlet type and specify a solution value~$\phi_D$
on the Dirichlet portion of the boundary~$\Gamma_D$
\begin{equation}
  \phi = \phi_D \quad \mathrm{on} \, \Gamma_D \,,
  \label{eqn:Laplace Dirichlet BC} 
\end{equation}
or of Neumann type and specify the solution gradient in normal direction, $g$,
on the Neumann portion of the boundary~$\Gamma_N$
\begin{equation}
  \nabla \phi \cdot {\bf{n}} = g \quad \mathrm{on} \, \Gamma_N
  \label{eqn:Laplace Dirichlet BC} 
\end{equation}
with $\Gamma = \Gamma_D \cup \Gamma_N$.

\subsubsection{Weak formulation:}
The corresponding weak form of \eqnref{eqn:Laplacesequation} is
\begin{equation}
  \gint{\Omega}{}{\diverg{\pbrac{\tensor{\sigma}\gradient{\phi}}}w}{\Omega}=0 \,,
  \label{eqn:Laplaceweakform}
\end{equation}
where $w$ is a suitable test function 
(For a definition of what constitutes a ``suitable'' test and trial function,
see Section~\ref{} - still has to be written).

Applying Green's theorem to \eqnref{eqn:Laplaceweakform} gives
\begin{equation}
 \gint{\Omega}{}{\diverg{\pbrac{\tensor{\sigma}\gradient{\phi}}}w}{\Omega}
 = -\gint{\Omega}{}{\dotprod{\pbrac{\tensor{\sigma}\gradient{\phi}}}{\gradient{w}}}{\Omega}
   +\gint{\Gamma}{}{\dotprod{\pbrac{\tensor{\sigma}\gradient{\phi}}}{\vect{n}}w}{\Gamma}
 = 0
\end{equation}
or
\begin{equation}
  \gint{\Omega}{}{\dotprod{\pbrac{\tensor{\sigma}\gradient{\phi}}}{\gradient{w}}}{\Omega}
  = \gint{\Gamma}{}{\dotprod{\pbrac{\tensor{\sigma}\gradient{\phi}}}{\vect{n}w}}{\Gamma} \,,
  \label{eqn:Laplaceweightedresidualform}
\end{equation}
where $\vect{n}$ is the unit outward normal vector to the boundary~$\Gamma$.

\subsubsection{Tensor notation:}
If we now consider the integrand of the left hand side of
\eqnref{eqn:Laplaceweightedresidualform} in tensorial form with covariant
derivatives indicated by a semi-colon in the index
(see \secref{subsec:function derivatives} for details) then
\begin{equation}
  \tensor{\sigma}\gradient{\phi} = \sigma^{i}_{j}\covarderiv{\phi}{i}
  \label{eqn:Laplace weak form factor1}
\end{equation}
and
\begin{equation}
  \gradient{w} = \covarderiv{w}{i}
  \label{eqn:Laplace weak form factor2}
\end{equation}

Now, Equations (\ref{eqn:Laplace weak form factor1}) and (\ref{eqn:Laplace weak form factor2})
represent vectors that are covariant and
therefore we must convert Equation (\ref{eqn:Laplace weak form factor1})
to a contravariant vector by 
multiplying by the contravariant metric tensor (from $\vect{i}$ to $\vect{x}$ 
coordinates; see Sections~\ref{sec:metric tensors} and \ref{sec:coordinate systems})
so that we can take the dot product. The final tensorial form is
\begin{equation}
  \dotprod{\pbrac{\tensor{\sigma}\gradient{\phi}}}{\gradient{w}} = G^{jk}\sigma^{i}_{j}\covarderiv{\phi}{i}\covarderiv{w}{k}
\end{equation}
and thus \eqnref{eqn:Laplaceweightedresidualform} becomes
\begin{equation}
  \gint{\Omega}{}{G^{jk}\sigma^{i}_{j}\covarderiv{\phi}{i}\covarderiv{w}{k}}{\Omega}
  = \gint{\Gamma}{}{\dotprod{\pbrac{\tensor{\sigma}\gradient{\phi}}}{\vect{n}w}}{\Gamma}
  \label{eqn:Laplaceweightedresidualtensorform}
\end{equation}

\subsubsection{Finite element formulation:}
We can now discretise the domain into finite elements \ie $\Omega=
\displaystyle{\bigcup_{e=1}^{E}}\Omega_{e}$, the left hand side of
\eqnref{eqn:Laplaceweightedresidualtensorform} becomes
\begin{equation}
  \dsum_{e=1}^{E}\gint{\Omega_{e}}{}{G^{jk}\sigma^{i}_{j}\covarderiv{\phi}{i}\covarderiv{w}{k}}{\Omega}
\end{equation}

The next step is to approximate the dependent variable, $\phi$, using basis
functions. To simplify this for different types of basis functions an
\emph{interpolation notation} is adopted. This
interpolation is such that $\gbfn{\alpha}{u}{\vect{\xi}}$ are the appropriate
basis functions for the type of element (\eg \bicubicherm, Hermite-sector,
\etc) and dimension of the problem (one, two or \threedal). For example if
$\vect{\xi}=\bbrac{\xi}$ and the element is a cubic Hermite element 
(\secref{sec:Hermitianbasisfunctions}) then
$\gbfn{\alpha}{u}{\vect{\xi}}$ are the cubic Hermite basis functions where
the local node number $\alpha$ ranges from $1$ to $2$ and $u$ ranges from $0$ to $1$. 
If, however, $\vect{\xi}=\bbrac{\xione,\xitwo}$ and the element is a \bicubicherm element
then $\alpha$ ranges from $1$ to $4$, $u$ ranges from $0$ to $3$ and
$\gbfn{\alpha}{u}{\vect{\xi}}$ is the appropriate basis function for the nodal
variable $\nodedof{\phi}{\alpha}{u}$. The nodal variables are defined as
follows: $\nodedof{\phi}{\alpha}{0}=\nodept{\phi}{\alpha}$,
$\nodedof{\phi}{\alpha}{1}=\delby{\nodept{\phi}{\alpha}}{s_{1}}$,
$\nodedof{\phi}{\alpha}{2}=\delby{\nodept{\phi}{\alpha}}{s_{2}}$,
$\nodedof{\phi}{\alpha}{3}=\deltwoby{\nodept{\phi}{\alpha}}{s_{1}}{s_{2}}$,
\etc Hence for the \bicubicherm element the interpolation function
$\gbfn{2}{3}{\vect{\xi}}$ multiplies the nodal variable
$\phi^{2}_{,3}=\deltwoby{\phi^{2}}{s_{1}}{s_{2}}$ and thus the
interpolation function is $\chbfn{2}{1}{\xione}\chbfn{1}{1}{\xitwo}$.  The
scale factors for the Hermite interpolation are handled by the introduction of
an interpolation scale factor $\gsf{\alpha}{u}$ defined as follows:
$\gsf{\alpha}{0}=1$, $\gsf{\alpha}{1}=\nsftwo{\alpha}{1}$, $\gsf{\alpha}{2}=
\nsftwo{\alpha}{2}$, $\gsf{\alpha}{3}=\nsftwo{\alpha}{1}\nsftwo{\alpha}{2}$,
\etc For Lagrangian basis functions the interpolation scale factors are all
one. The general form of the interpolation notation for $\phi$ is then
\begin{equation}
  \fnof{\phi}{\vect{\xi}}=\gbfn{\alpha}{u}{\vect{\xi}}\nodedof{\phi}
  {\alpha}{u}\gsf{(\alpha)}{(u)}
  \label{eqn:phiinterpolation}
\end{equation}

\subsubsection{Spatial integration:}
When dealing with integrals a similar interpolation notation is adopted in
that $d\vect{\xi}$ is the appropriate infinitesimal for the dimension of the
problem. The limits of the integral are also written in bold font and indicate
the appropriate number of integrals for the dimension of the problem.  For
example if $\vect{\xi}=\bbrac{\xione,\xitwo}$ then
$\gint{\vect{0}}{\vect{1}}{x}
{\vect{\xi}}=\giint{0}{1}{0}{1}{x}{\xione}{\xitwo}$, but if $\vect{\xi}=
\bbrac{\xione,\xitwo,\xithree}$ then $\gint{\vect{0}}{\vect{1}}{x}
{\vect{\xi}}=\dintl{0}{1}\!\dintl{0}{1}\!\dintl{0}{1}\,x\,d\xione d\xitwo
d\xithree$.

In order to integrate in $\vect{\xi}$ coordinates we must convert the spatial
derivatives to derivatives with respect to $\vect{\xi}$. Using the tensor
transformation equations for a covariant vector we have
\begin{equation}  
  \covarderiv{\phi}{i}=\delby{\xi^{r}}{x^{i}}\covarderiv{\phi}{r}=\delby{\xi^{r}}{x^{i}}\delby{\phi}{\xi^{r}}
\end{equation}
and 
\begin{equation}
  \covarderiv{w}{k}=\delby{\xi^{s}}{x^{k}}\covarderiv{w}{s}=\delby{\xi^{s}}{x^{k}}\delby{w}{\xi^{s}}
\end{equation}

As we only know $\tensor{\tilde{\sigma}}$, the conductivity in the
$\vect{\nu}$ (fibre) coordinate system rather than $\tensor{\sigma}$, the
conductivity in the $\vect{x}$ (geometric) coordinate system, we must transform the mixed
tensor ${\tilde{\sigma}}^{a}_{.b}$ from $\vect{\nu}$ to $\vect{x}$ coordinates. However, as the
fibre coordinate system is defined in terms of $\vect{\xi}$ coordinates we
first transform the conductivity tensor from $\vect{\nu}$ to $\vect{\xi}$
coordinates \ie
\begin{equation}
  \sigma^{p}_{.q}=\delby{\xi^{p}}{\nu^{a}}\delby{\nu^{b}}{\xi^{q}}{\tilde{\sigma}}^{a}_{.b}
\end{equation}
with $.b$ indicating that $b$ is the ``second'' index (\secref{subsec:summation notation}),
and then transform the conductivity from $\vect{\xi}$ coordinates to
$\vect{x}$ coordinates \ie
\begin{equation}
  \begin{split}
    \sigma^{i}_{.j} &= \delby{x^{i}}{\xi^{p}}\delby{\xi^{q}}{x^{j}}{\sigma}^{p}_{.q} \\
    &= \delby{x^{i}}{\xi^{p}}\delby{\xi^{q}}{x^{j}}\delby{\xi^{p}} 
    {\nu^{a}}\delby{\nu^{b}}{\xi^{q}}{\tilde{\sigma}}^{a}_{.b}
  \end{split}
\end{equation}

Using an interpolated dependent variable and a Galerkin finite element
approach (where the weighting functions are chosen to be the interpolating
basis functions \ie $w=\gbfn{\beta}{v}{\vect{\xi}}\gsf{(\beta)}{(v)}$) yields
\begin{equation}
  \dsum_{e=1}^{E}\gint{\vect{0}}{\vect{1}}{G^{jk}\delby{x^{i}}{\xi^{p}}
    \delby{\xi^{q}}{x^{j}}\delby{\xi^{p}} {\nu^{a}}
    \delby{\nu^{b}}{\xi^{q}}{\tilde{\sigma}}^{a}_{.b}\delby{\xi^{r}}{x^{i}}
    \delby{\pbrac{\gbfn{\alpha}{u}{\vect{\xi}}\nodedof{\phi}{\alpha}{u}\gsf{(\alpha)}{(u)}}}{\xi^{r}}
    \delby{\xi^{s}}{x^{k}}\delby{\pbrac{\gbfn{\beta}{v}{\vect{\xi}}\gsf{(\beta)}{(v)}}}{\xi^{s}}
    \abs{\fnof{\matr{J}}{\vect{\xi}}}}{\vect{\xi}}
\end{equation}
where $\fnof{\matr{J}}{\vect{\xi}}$ is the \emph{Jacobian} of the
transformation from the integration $\vect{x}$ to $\vect{\xi}$ coordinates.

Taking the fixed nodal degrees-of-freedom and scale factors outside the integral we have
\begin{equation}
  \dsum_{e=1}^{E}\nodedof{\phi}{\alpha}{u}\gsf{(\alpha)}{(u)}\gsf{(\beta)}{(v)}
  \gint{\vect{0}}{\vect{1}}{G^{jk}\delby{x^{i}}{\xi^{p}}
    \delby{\xi^{q}}{x^{j}}\delby{\xi^{p}} {\nu^{a}}
    \delby{\nu^{b}}{\xi^{q}}{\tilde{\sigma}}^{a}_{.b}\delby{\xi^{r}}{x^{i}}
    \delby{\gbfn{\alpha}{u}{\vect{\xi}}}{\xi^{r}}
    \delby{\xi^{s}}{x^{k}}\delby{\gbfn{\beta}{v}{\vect{\xi}}}{\xi^{s}}
    \abs{\fnof{\matr{J}}{\vect{\xi}}}}{\vect{\xi}}
  \label{eqn:elementalfemlhs1}
\end{equation}
or
\begin{equation}
  \dsum_{e=1}^{E}\nodedof{\phi}{\alpha}{u}\gsf{(\alpha)}{(u)}\gsf{(\beta)}{(v)}
  \gint{\vect{0}}{\vect{1}}{\delby{\gbfn{\alpha}{u}{\vect{\xi}}}{\xi^{r}}\delby{\gbfn{\beta}{v}{\vect{\xi}}}{\xi^{s}}\gamma^{rs}
    \abs{\fnof{\matr{J}}{\vect{\xi}}}}{\vect{\xi}}
  \label{eqn:elementalfemlhs2}
\end{equation}
where
\begin{equation}
  \gamma^{rs}=G^{jk}\delby{x^{i}}{\xi^{p}}
    \delby{\xi^{q}}{x^{j}}\delby{\xi^{p}} {\nu^{a}}
    \delby{\nu^{b}}{\xi^{q}}{\tilde{\sigma}}^{a}_{.b}\delby{\xi^{r}}{x^{i}}\delby{\xi^{s}}{x^{k}}
\end{equation}

Note that for Laplace's equation with no conductivity or fibre fields we have
\begin{equation}
  \gamma^{rs}=G^{ik}\delby{\xi^{r}}{x^{i}}\delby{\xi^{s}}{x^{k}}
\end{equation}
and that for rectangular-Cartesian coordinates systems
$G^{ik}=\contrakronecker{i}{k}$ and thus $i=k$. This now gives
\begin{equation}
  \gamma^{rs}=\delby{\xi^{r}}{x^{i}}\delby{\xi^{s}}{x^{i}}=g^{rs}
  \label{eqn:contrametrictensor}
\end{equation}
where $g^{rs}$ is the \emph{contravariant metric tensor} from $\vect{x}$ to
$\vect{\xi}$ coordinates. It may thus be helpful to think about $\gamma^{rs}$
as a scaled/transformed contravariant metric tensor.

\subsubsection{Coded OpenCMISS formulation:}
Finally, we use the Gaussian quadrature rule (see Section~??? - still to be written), 
usually stated as a weighted sum of function values at specified Gauss points within the domain of integration. An $n$-point Gaussian quadrature rule yields an exact result for polynomials of degree $2n-1$ or less by a suitable choice of the points $x_i$ and weights $g_i$ for $i = 1,...,n$. Consequently, the formulation implemented can be derived from Equation (\ref{eqn:elementalfemlhs2}) as: 
\begin{equation}
  \boxed{
  \dsum_{e=1}^{E}\nodedof{\phi}{\alpha}{u}\gsf{(\alpha)}{(u)} \gsf{(\beta)}{(v)}
  \dsum_{i=1}^{n}
  \left(\delby{\gbfn{\alpha}{u}{\vect{\xi}}}{\xi^{r}}
  \delby{\gbfn{\beta}{v}{\vect{\xi}}}{\xi^{s}}\gamma^{rs}\abs{\fnof{\matr{J}}{\vect{\xi}}}
  \right)(x_i)g_i
  }
  \label{eqn:Laplaceweightedresidualtensorform in OpenCMISS}
\end{equation}


\newpage
\subsubsection{OpenCMISS implementation}

For exemplification, we briefly discuss here a code snippet
that evaluates the expression~(\ref{eqn:Laplaceweightedresidualtensorform in OpenCMISS}):

\lstset{ %
language=Fortran,               % choose the language of the code
basicstyle=\footnotesize,       % the size of the fonts that are used for the code
numbers=left,                   % where to put the line-numbers
numberstyle=\footnotesize,      % the size of the fonts that are used for the line-numbers
stepnumber=1,                   % the step between two line-numbers. If it's 1 each line 
                                % will be numbered
numbersep=5pt,                  % how far the line-numbers are from the code
backgroundcolor=\color{white},  % choose the background color. You must add \usepackage{color}
showspaces=false,               % show spaces adding particular underscores
showstringspaces=false,         % underline spaces within strings
showtabs=false,                 % show tabs within strings adding particular underscores
frame=single,	                % adds a frame around the code
tabsize=2,	                % sets default tabsize to 2 spaces
captionpos=b,                   % sets the caption-position to bottom
breaklines=true,                % sets automatic line breaking
breakatwhitespace=false,        % sets if automatic breaks should only happen at whitespace
title=\lstname,                 % show the filename of files included with \lstinputlisting;
                                % also try caption instead of title
escapeinside={\%*}{*)},         % if you want to add a comment within your code
morekeywords={*,...}            % if you want to add more keywords to the set
}




{\footnotesize\tt
\begin{lstlisting}
SUBROUTINE LAPLACE_EQUATION_FINITE_ELEMENT_CALCULATE(EQUATIONS_SET,ELEMENT_NUMBER, ERR,ERROR,*)
  DO ng=1,QUADRATURE_SCHEME%NUMBER_OF_GAUSS
    RWG=EQUATIONS%INTERPOLATION%GEOMETRIC_INTERP_POINT_METRICS(FIELD_U_VARIABLE_TYPE)%PTR%JACOBIAN*QUADRATURE_SCHEME%GAUSS_WEIGHTS(ng)
    mhs=0          
    !Loop over element rows
    DO mh=1,FIELD_VARIABLE%NUMBER_OF_COMPONENTS
      DO ms=1,DEPENDENT_BASIS%NUMBER_OF_ELEMENT_PARAMETERS
        mhs=mhs+1
        nhs=0
        IF(EQUATIONS_MATRIX%UPDATE_MATRIX) THEN
          !Loop over element columns
          DO nh=1,FIELD_VARIABLE%NUMBER_OF_COMPONENTS
            DO ns=1,DEPENDENT_BASIS%NUMBER_OF_ELEMENT_PARAMETERS
              nhs=nhs+1
              DO ni=1,DEPENDENT_BASIS%NUMBER_OF_XI
                PGMSI(ni)=QUADRATURE_SCHEME%GAUSS_BASIS_FNS(ms,PARTIAL_DERIVATIVE_FIRST_DERIVATIVE_MAP(ni),ng)
                PGNSI(ni)=QUADRATURE_SCHEME%GAUSS_BASIS_FNS(ns,PARTIAL_DERIVATIVE_FIRST_DERIVATIVE_MAP(ni),ng)
              ENDDO !ni
              SUM=0.0_DP
              DO mi=1,DEPENDENT_BASIS%NUMBER_OF_XI
                DO ni=1,DEPENDENT_BASIS%NUMBER_OF_XI
                  SUM=SUM+PGMSI(mi)*PGNSI(ni)*EQUATIONS%INTERPOLATION% &
                    & GEOMETRIC_INTERP_POINT_METRICS(FIELD_U_VARIABLE_TYPE)%PTR%GU(mi,ni)
                ENDDO !ni
              ENDDO !mi
              EQUATIONS_MATRIX%ELEMENT_MATRIX%MATRIX(mhs,nhs)=EQUATIONS_MATRIX%ELEMENT_MATRIX%MATRIX(mhs,nhs)+SUM*RWG
            ENDDO !ns
          ENDDO !nh
        ENDIF
      ENDDO !ms
    ENDDO !mh
  ENDDO !ng
\end{lstlisting}
}


% * Still explain that we loop over the rows of the element matrix,
%   and the loop is composed of a loop over number of components and element parameters.

\begin{tabular}{|rp{0.9\textwidth}|}
  \hline
  Line & Explanation \\
  & \\
  2 & Loop over the number of Gauss points\\
  3 & Weight factor = Jacobian of the mapping * quadrature weight\\
  6 & Loop ({\bf rows}) over the number of variable components 
      (For the scalar case of Laplace, this number is 1 (scalar potential),
      whereas for a system like 3D incompressible Navier-Stokes, 
      this number is 4 (3 velocity components and pressure))\\
  7 & Loop ({\bf rows}) over the number of element parameters (= the number of interpolation functions 
      associated with this element) \\
  12 & Loop ({\bf columns}) over the number of variable components\\ 
  13 & Loop ({\bf columns}) over the number of element parameters\\
  15 & Loop over the element $\xi$-directions\\
  20 & Loop over the element $\xi$-directions\\
  21 & Loop over the element $\xi$-directions\\
  26 & Add contribution to element matrix entry {\tt (mhs,nhs)}\\
  \hline
\end{tabular}



\clearpage
%\input{AnalyticSolutions/ClassicalFieldClass/PoissonEquation.tex}
\clearpage
%\input{AnalyticSolutions/ClassicalFieldClass/HelmholtzEquation.tex}
\clearpage
%\input{AnalyticSolutions/ClassicalFieldClass/WaveEquation.tex}
\clearpage
\subsection{Diffusion Equation} 

\subsubsection{One Dimensional Analytic Function 1}

From \url{http://eqworld.ipmnet.ru/en/solutions/lpde/lpde101.pdf} we have for the \oned diffusion equation
\begin{equation}
  \delby{u}{t}=a\deltwosqby{u}{x}
\end{equation}
the analytic solution
\begin{equation}
  \fnof{u}{x,t}=Ae^{-a\mu^{2}t}\cos(\mu x + B) + C
\end{equation}

Now, \OpenCMISS solves diffusion equations of the form
\begin{equation}
  \delby{u}{t}+k\deltwosqby{u}{x}=0
  \label{eqn:OpenCMISSDiffusionEquationOneDim}
\end{equation}
and therefore $a=-k$. Choosing $\mu=\dfrac{2\pi}{L}$ gives the
\OpenCMISS diffusion equation one dimensional analytic function 1, namely
\begin{equation}
  \fnof{u}{x,t}=Ae^{\frac{4\pi^{2}kt}{L^{2}}}\cos(\frac{2\pi x}{L} + B) + C
  \label{eqn:AnalyticDiffusionEquationOneDim1}
\end{equation}

To prove the analytic solution we differentiate
\eqnref{eqn:AnalyticDiffusionEquationOneDim1} to give
\begin{align}
    \delby{u}{t} &=
    \dfrac{4\pi^{2}k}{L^{2}}Ae^{\frac{4\pi^{2}kt}{L^{2}}}\cos(\frac{2\pi x}{L} + B) \\
    \delby{u}{x} &=
    \dfrac{-2\pi}{L}Ae^{\frac{4\pi^{2}kt}{L^{2}}}\sin(\frac{2\pi x}{L} + B) \\
    \deltwosqby{u}{x} &=
    \dfrac{2\pi}{L}\dfrac{-2\pi}{L}Ae^{\frac{4\pi^{2}kt}{L^{2}}}\cos(\frac{2\pi x}{L} + B) \\
    & = \dfrac{-4\pi^{2}}{L^{2}}Ae^{\frac{4\pi^{2}kt}{L^{2}}}\cos(\frac{2\pi x}{L} + B)
  \label{eqn:AnalyticDiffusionEquationOneDim1Derivatives}
\end{align}

Substiting \eqnref{eqn:AnalyticDiffusionEquationOneDim1Derivatives} into \eqnref{eqn:OpenCMISSDiffusionEquationOneDim} gives
\begin{align}
  \delby{u}{t}+k\deltwosqby{u}{x} &=
  \dfrac{4\pi^{2}k}{L^{2}}Ae^{\frac{4\pi^{2}kt}{L^{2}}}\cos(\frac{2\pi x}{L}
  + B) +
  k\dfrac{-4\pi^{2}}{L^{2}}Ae^{\frac{4\pi^{2}kt}{L^{2}}}\cos(\frac{2\pi x}{L}
  + B) \\
  &= 0
\end{align}

The analytic field component definitions in \OpenCMISS are shown in \tabref{tab:OpenCMISSAnalyticFieldDiffusionEquationOneDim1}.

\begin{table}[htb] \centering
  \begin{tabular}{|c|c|} \hline
    \emph{Analytic constant} & \emph{Analytic field component} \\ \hline \hline
    $A$ & $1$ \\ 
    $B$ & $2$ \\
    $C$ & $3$ \\
    $L$ & $4$ \\ \hline
  \end{tabular}
  \caption{\OpenCMISS analytic field components for the \oned diffusion equation
    analytic function 1.}
  \label{tab:OpenCMISSAnalyticFieldDiffusionEquationOneDim1}
\end{table}

\subsubsection{One Dimensional Quadratic Source Analytic Function 1}

From \url{http://eqworld.ipmnet.ru/en/solutions/npde/npde1104.pdf} we have for
the \oned polynomial source diffusion equation
\begin{equation}
  \delby{u}{t}=\deltwosqby{u}{x}+qw+rw^{m}
\end{equation}
the analytic solution
\begin{equation}
  \fnof{u}{x,t}=\sqbrac{\pm\beta+Ae^{\pbrac{\lambda t \pm\mu x}}}^\frac{2}{1-m}
\end{equation}
where
\begin{equation}
  \beta=\sqrt{\dfrac{-r}{q}}
\end{equation}
and
\begin{equation}
  \lambda=\dfrac{q\pbrac{1-m}\pbrac{m+3}}{2\pbrac{m+1}}
\end{equation}
and 
\begin{equation}
  \mu=\sqrt{\dfrac{q\pbrac{1-m}^{2}}{2\pbrac{m+1}}}
\end{equation}

Now, \OpenCMISS solves quadratic source diffusion equations of the form
\begin{equation}
  \delby{u}{t}+k\deltwosqby{u}{x}+a+bu+cu^{2}=0
  \label{eqn:OpenCMISSQuadraticSourceDiffusionEquationOneDim}
\end{equation}
and therefore $k=-1$, $m=2$, $a=0$, $q=-b$ and $r=-c$. Choosing the positive wave gives the
\OpenCMISS quadratic source diffusion equation one dimensional analytic function 1, namely
\begin{equation}
  \fnof{u}{x,t}=\dfrac{1}{\sqbrac{\beta+Ae^{\pbrac{\lambda t + \mu x}}}^{2}}
  \label{eqn:AnalyticQuadraticSourceDiffusionEquationOneDim1}
\end{equation}
where
\begin{equation}
  \beta=\sqrt{\dfrac{-c}{b}}
\end{equation}
and
\begin{equation}
  \lambda=\dfrac{5b}{6}
\end{equation}
and 
\begin{equation}
  \mu=\sqrt{\dfrac{-b}{6}}
\end{equation}

To prove the analytic solution we differentiate
\eqnref{eqn:AnalyticQuadraticSourceDiffusionEquationOneDim1} to give 
\begin{align}
  \delby{u}{t} &= \dfrac{2\lambda A\pbrac{\lambda t+\mu x}e^{\pbrac{\lambda t+\mu
        x}}}{\sqbrac{\beta+Ae^{\pbrac{\lambda t + \mu x}}}^{3}} \\
  \delby{u}{x} &= \dfrac{2\mu A\pbrac{\lambda t+\mu x}e^{\pbrac{\lambda t+\mu
        x}}}{\sqbrac{\beta+Ae^{\pbrac{\lambda t + \mu x}}}^{3}} \\
  \deltwosqby{u}{x} &=
  \label{eqn:AnalyticDiffusionEquationOneDim1Derivatives}
\end{align}

The analytic field component definitions in \OpenCMISS are shown in \tabref{tab:OpenCMISSAnalyticFieldQuadraticFieldDiffusionEquationOneDim1}.

\begin{table}[htb] \centering
  \begin{tabular}{|c|c|} \hline
    \emph{Analytic constant} & \emph{Analytic field component} \\ \hline \hline
    $A$ & $1$ \\ \hline
  \end{tabular}
  \caption{\OpenCMISS analytic field components for the \oned quadratic
    source diffusion equation analytic function 1.}
  \label{tab:OpenCMISSAnalyticFieldQuadraticSoureDiffusionEquationOneDim1}
\end{table}

\subsubsection{One Dimensional Exponential Source Analytic Function 1}

From \url{http://eqworld.ipmnet.ru/en/solutions/npde/npde1105.pdf} we have for
the \oned exponential source diffusion equation
\begin{equation}
  \delby{u}{t}=\deltwosqby{u}{x}+q+re^{\lambda u}
\end{equation}
the analytic solution
\begin{equation}
  \fnof{u}{x,t}=\dfrac{-2}{\lambda}\ln\sqbrac{\pm\beta+Ae^{\pbrac{\pm\mu x - \frac{1}{2}q \lambda t}}}
\end{equation}
where
\begin{equation}
  \beta=\sqrt{\dfrac{-r}{q}}
\end{equation}
and 
\begin{equation}
  \mu=\sqrt{\dfrac{q\lambda}{2}}
\end{equation}

Now, \OpenCMISS solves diffusion equations of the form
\begin{equation}
  \delby{u}{t}+k\deltwosqby{u}{x}+a+be^{cx}=0
  \label{eqn:OpenCMISSExponentialSourceDiffusionEquationOneDim}
\end{equation}
and therefore $k=-1$, $q=-a$ and $r=-b$. Choosing the positive wave gives the
\OpenCMISS exponential source diffusion equation one dimensional analytic function 1, namely
\begin{equation}
  \fnof{u}{x,t}=\dfrac{-2}{c}\ln\sqbrac{\beta+Ae^{\mu\pbrac{x - \mu t}}}
  \label{eqn:AnalyticExponentialSourceDiffusionEquationOneDim1}
\end{equation}
where
\begin{equation}
  \beta=\sqrt{\dfrac{-b}{a}}
\end{equation}
and 
\begin{equation}
  \mu=\sqrt{\dfrac{-ac}{2}}
\end{equation}

To prove the analytic solution we differentiate
\eqnref{eqn:AnalyticExponentialSourceDiffusionEquationOneDim1} to give .....

The analytic field component definitions in \OpenCMISS are shown in \tabref{tab:OpenCMISSAnalyticFieldExponentialFieldDiffusionEquationOneDim1}.

\begin{table}[htb] \centering
  \begin{tabular}{|c|c|} \hline
    \emph{Analytic constant} & \emph{Analytic field component} \\ \hline \hline
    $A$ & $1$ \\ \hline
  \end{tabular}
  \caption{\OpenCMISS analytic field components for the \oned exponential
    source diffusion equation analytic function 1.}
  \label{tab:OpenCMISSAnalyticFieldExponentialSoureDiffusionEquationOneDim1}
\end{table}


\clearpage

\subsection{Hamilton-Jacobi Equation}

\subsubsection{Anisotropic Eikonal Solver}

Available analyticsolutions in the Hamilton-Jacobi Equations of Anisotropic Eikonal solver are as below:


\textbf{Analytic Solution 1:} Solving with $a_{ij}=\delta_{ij}$ and function $f(x,y)=1$. The computational domain is $\Omega=[-1,1]^2$, and $\partial\Omega$ is a source point at the origin with coordinates $(0,0)$. The exact solution is $$\phi(x,y)=\sqrt{x^2+y^2}$$

\begin{figure}[h]
  \centering
    \includegraphics[width=0.75\textwidth]{../latex/epsfiles/HJE/FigEx1Res.eps}
  \caption{Analytic Solution 1 results.}
  \label{fig:Ex1}
\end{figure}


\textbf{Analytic Solution 2:} Solving with $a_{ij}=\delta_{ij}$ and $$f(x,y)=2\pi \sqrt{10[\cos{(2\pi x)}\sin{(2\pi y)}]^2+[\sin{(2\pi x)}\cos{(2\pi y)}]^2}$$ The boundary is given by $\partial\Omega=\{(\frac{1}{4},\frac{1}{4}),(\frac{3}{4},\frac{3}{4}),(\frac{3}{4},\frac{1}{4}),(\frac{1}{4},\frac{3}{4}),(\frac{1}{2},\frac{1}{2})\}=\{1,1,-1,-1,0\}$, which consists of five isolated points. The computational domain is $\Omega=[0,1]^2$. $\phi(x,y)=0$ is prescribed at the boundary of the unit square. The exact solution for this case is $$\phi(x,y)=\sin{(2\pi x)} \sin{(2\pi y)}$$

\begin{figure}[h]
  \centering
    \includegraphics[width=0.75\textwidth]{../latex/epsfiles/HJE/FigEx2Res.eps}
  \caption{Analytic Solution 2 results.}
  \label{fig:Ex2}
\end{figure}


\textbf{Analytic Solution 3:} Solving with $a_{ij}=\delta_{ij}$ and $$f(x,y,z)=\frac{\pi}{2} \sqrt{\sin^2{(\pi + \frac{\pi}{2}x)}+\sin^2{(\pi + \frac{\pi}{2}y)}+\sin^2{(\pi + \frac{\pi}{2}z)}}$$ The computational domain is $\Omega=[-1,1]$ in $x$ and $y$ direction and $\Omega=[-1,1]^3$ at $z$ direction. The inflow boundary $\partial\Omega$ is a source at the origin $(0,0,0)$. The exact solution for this problem is $$\phi(x,y,z)=\cos{(\pi + \frac{\pi}{2}x})+\cos{(\pi + \frac{\pi}{2}y})+\cos{(\pi + \frac{\pi}{2}z})$$

\begin{figure}[h]
  \centering
    \includegraphics[width=0.75\textwidth]{../latex/epsfiles/HJE/FigEx3Res.eps}
  \caption{Analytic Solution 3 results.}
  \label{fig:Ex3}
\end{figure}


\textbf{Two Element Example:} Solving with anisotropic case for two element domain with constant function $f(x,y)=1$ along fiber directions and $f(x,y)=10$ in plane of isotropy.

\begin{figure}[h]
  \centering
    \includegraphics[width=0.75\textwidth]{../latex/epsfiles/HJE/FigEx4Res.eps}
  \caption{Two Element Example results.}
  \label{fig:Ex4}
\end{figure}


\subsubsection{Geodesic Solver}

\noindent The analyticsolutions for Geodesic solver which is included in the library is:

\textbf{GeodesicEx1} Solving within a cubic mesh domain, the aim is to find the geodesic path between one corner of the cube and one point on a face of the cube using this algorithm.
Depending on the application the Geodesic nodes list and Geodesic path lengths is available.

\begin{figure}[h]
  \centering
    \includegraphics[width=0.4\textwidth]{../latex/epsfiles/HJE/FigGeodesicExRes.eps}
  \caption{Geodesic path from the example.}
  \label{fig:GeodesicEx}
\end{figure}

\clearpage

%\input{AnalyticSolutions/ClassicalFieldClass/AdvectionDiffusionEquation.tex}
\clearpage
%\input{AnalyticSolutions/ClassicalFieldClass/ReactionDiffusionEquation.tex}
\clearpage
%\input{AnalyticSolutions/ClassicalFieldClass/BiharmonicEquation.tex}


\section{Elasticity Class}
\clearpage
%\input{AnalyticSolutions/ElasticityClass/LinearElasticity.tex}
\clearpage
%\input{AnalyticSolutions/ElasticityClass/FiniteElasticity.tex}
\clearpage

\section{Fluid Mechanics Class}
\clearpage
%\input{AnalyticSolutions/FluidMechanicsClass/StokesEquation.tex}
\clearpage
%\input{AnalyticSolutions/FluidMechanicsClass/DarcyEquation.tex}
\clearpage
%\input{AnalyticSolutions/FluidMechanicsClass/NavierStokesEquation.tex}
\clearpage
%\input{AnalyticSolutions/FluidMechanicsClass/PressurePoissonEquation.tex}
\clearpage
\subsection{Burgers's Equation}

\subsubsection{Governing equations:}

For a given velocity $\fnof{u}{x,t}$ and a kinematic viscosity of $\nu$, Burgers's equation
in one dimension is
\begin{equation}
  \delby{u}{t}+u\delby{u}{x}=\nu\deltwosqby{u}{x}
  \label{eqn:Burgersequation}
\end{equation}

The general form of this equation in \OpenCMISS is
\begin{equation}
  \fnof{a}{\vect{x}}\delby{\fnof{\vect{u}}{\vect{x},t}}{t}+\fnof{b}{\vect{x}}\laplacian{\fnof{\vect{u}}{\vect{x},t}}+
  \fnof{c}{\vect{x}}\dotprod{\fnof{\vect{u}}{\vect{x},t}}{\gradient{\fnof{\vect{u}}{\vect{x},t}}}=\vect{0}
  \label{eqn:GeneralFormBurgersequation}
\end{equation}
where $\fnof{\vect{u}}{\vect{x},t}$ is the velocity vector and
$\fnof{a}{\vect{x}}$, $\fnof{b}{\vect{x}}$ and $\fnof{c}{\vect{x}}$ are
material parameters. The standard form of Burgers's equation can be found with
$a=1$, $b=-\nu$ and $c=1$.

\subsubsection{Weak formulation:}

The corresponding weak form of \eqnref{eqn:GeneralFormBurgersequation} can be
found by integrating over the domain with a test function \ie
\begin{equation}
  \gint{\tau}{}{\sqbrac{\gint{\Omega}{}{\pbrac{a\delby{\vect{u}}{t}+
          b\laplacian{\vect{u}}+
          c\dotprod{\vect{u}}{\gradient{\vect{u}}}}\vect{w}}{\Omega}}v}{\tau}=0
  \label{eqn:Burgersweakform1}
\end{equation}
where $\vect{w}$ are suitable spatial test functions and $v$ is a suitable temporal
test funtion.

Applying the divergence theorem in \eqnref{eqn:DivergenceTheormScalarVector} to \eqnref{eqn:Burgersweakform} gives
\begin{equation}
  \gint{\tau}{}{\sqbrac{\gint{\Omega}{}{\pbrac{a\delby{\vect{u}}{t}}\vect{w}}{\Omega}-
      \gint{\Omega}{}{b\dotprod{\gradient{\vect{u}}}{\gradient{\vect{w}}}}{\Omega}
      +\gint{\Gamma}{}{\pbrac{b\dotprod{\gradient{\vect{u}}}{\vect{n}}}\vect{w}}{\Gamma}+
      \gint{\Omega}{}{\pbrac{c\dotprod{\vect{u}}{\gradient{\vect{u}}}}\vect{w}}{\Omega}}v}{\tau}=0
  \label{eqn:Burgersweakform2}
\end{equation}

\subsubsection{Tensor notation:}

\Eqnref{eqn:Burgersweakform2} can be written in tensor notation as
\begin{equation}
  \gint{\tau}{}{\sqbrac{\gint{\Omega}{}{a\dot{u_{i}}w_{i}}{\Omega}-
      \gint{\Omega}{}{bG^{jk}\covarderiv{u_{i}}{j}\covarderiv{w_{i}}{k}}{\Omega}+
      \gint{\Gamma}{}{bG^{jk}\covarderiv{u_{i}}{k}n_{j}w_{i}}{\Gamma} +
      \gint{\Omega}{}{cG^{jk}u_{j}\covarderiv{u_{i}}{k}w_{i}}{\Omega}}v}{\tau}=0
  \label{eqn:Burgerstensorform1}
\end{equation}
where $G^{jk}$ is the contravariant metric tensor for the spatial coordinates.

As \eqnref{eqn:Burgerstensorform1} must hold for an arbitrary choice of $v$
then we can write the expression within the square brackets of
\eqnref{eqn:Burgerstensorform1} as
\begin{multline}
  \gint{\Omega}{}{a\dot{u_{i}}w_{i}}{\Omega}-
  \gint{\Omega}{}{bG^{jk}\pbrac{\partialderiv{u_{i}}{j}-\christoffel{i}{j}{l}u_{l}}\partialderiv{w_{i}}{k}}{\Omega}+\\
  \gint{\Omega}{}{cG^{jk}u_{j}\pbrac{\partialderiv{u_{i}}{k}-\christoffel{i}{k}{l}u_{l}}w_{i}}{\Omega} =
  -\gint{\Gamma}{}{bG^{jk}\pbrac{\partialderiv{u_{i}}{k}-\christoffel{i}{k}{l}u_{l}}n_{j}w_{i}}{\Gamma} 
  \label{eqn:Burgerstensorform2}
\end{multline}

\subsubsection{Finite element formulation:}
We can now discretise the domain into finite elements \ie $\Omega=
\displaystyle{\bigcup_{e=1}^{E}}\Omega_{e}$
If we now interpolate $\vect{u}$ using basis functions and use a Galerkin
finite elements we have
\begin{align}
  \fnof{u_{i}}{\vect{\xi}} &=
  \idxgbfn{i}{\alpha}{f}{\vect{\xi}}\fnof{\idxnodedof{u}{i}{\alpha}{f}}{t}\gsf{(\alpha)}{(f)}
  \\
  \fnof{w_{i}}{\vect{\xi}} &= \idxgbfn{i}{\beta}{f}{\vect{\xi}}
\end{align}

The equation within the square brackets of \eqnref{eqn:Burgerstensorform1} is
of the form
\begin{equation}
  \matr{C}\fnof{\dot{\vect{u}}}{t}+\matr{K}\fnof{\vect{u}}{t}+\fnof{\vect{g}}{\fnof{\vect{u}}{t}}=\fnof{\vect{f}}{t}
\end{equation}
where
\begin{align}
  \matr{C} &=
\end{align}


\section{Electromechanics Class}
\clearpage
%\input{AnalyticSolutions/ElectroMechanicsClass/ElectrostaticEquations.tex}
\clearpage
%\input{AnalyticSolutions/ElectroMechanicsClass/MagnetostaticEquations.tex}
\clearpage
%\input{AnalyticSolutions/ElectroMechanicsClass/MaxwellEquations.tex}



\section{Bioelectrics Class}
\clearpage
%\input{AnalyticSolutions/BioElectricsClass/BidomainEquation.tex}
\clearpage
%
\subsection{Monodomain Equation}

Under certain conditions the Biodomain equations given in
\eqnref{eqn:bidomain1} and \eqnref{eqn:bidomain2} can be simplified to 

\begin{equation}
  A_{m}C_{m}\delby{V_{m}}{t}-\diverg{\pbrac{\tensor{\sigma}_{m}\grad{V_{m}}}}=-A_{m}I_{ion}+i_{i}
  \label{eqn:monodomain}
\end{equation}

where $A_{m}$ is the surface to volume ratio for the cell, $C_{m}$ is
the specific membrance capacitance, $I_{ion}$ is the ionic current, $i_{i}$ is the
intracellular stimulus current and $\tensor{\sigma}_{m}$ is the
conductivity tensor given by

\begin{equation}
  \tensor{\sigma}_{m}=\dfrac{\tensor{\sigma}_{i}\tensor{\sigma}_{e}}{\tensor{\sigma}_{i}+\tensor{\sigma}_{e}}
  \label{eqn:monodomainconductivity}
\end{equation}

Rearrangign \eqnref{eqn:monodomain} to give
\begin{equation}
  A_{m}C_{m}\delby{V_{m}}{t}=\dfrac{\pbrac{\diverg{\pbrac{\tensor{\sigma}_{m}\grad{V_{m}}}}-I_{ion}+i_{i}}}{}
\end{equation}

%\input{AnalyticSolutions/BioElectricsClass/MonodomainEquation.tex}

\section{Modal Class}
\clearpage

\section{Fitting Class}
\clearpage
%\subsection{Linear Fitting}

\subsubsection{Problem formulation}

The \emph{geometric fitting problem} is, given a set of data defining some
geometry, find the nodal positions and derivatives that, for a given mesh,
minimises the error in the mesh approximation to the data. Hence before a
fitting problem can be formulated the error in the mesh approximation must be
defined.

Consider a set of rectangular Cartesian data with geometric locations
$\dpt{\vect{z}}{d}$, $d=1,\ldots,D$. For each data point a point on the mesh
that has the smallest distance to that data point can be found. This point is
termed the \emph{orthogonal projection} of the data point onto the mesh. This
point is specified by a local element coordinate, $\dpt{\vect{\xi}}{d}$, in
the element of projection, $\dpt{\varepsilon}{d}$. The geometric location of
this point, $\fnof{\vect{z}}{\dpt{\vect{\xi}}{d}}$, is found from
interpolation \ie
\begin{equation}
  \fnof{\vect{z}}{\dpt{\vect{\xi}}{d}}=\gbfn{\alpha}{u}{\dpt{\vect{\xi}}{d}}
  \npt{\vect{x}}{\alpha}_{,u}\gsf{(\alpha)}{(u)}
  \label{eqn:dataprojectionpos}
\end{equation}
A schematic of this projection is shown in \figref{fig:dataprojection}.  To
calculate $\dpt{\vect{\xi}}{d}$ a non-linear iterative calculation is
required, details of which are given later in this subsection.

\pstexfigure{figs/TorsoModel/dataprojection.pstex}{Schematic of an orthogonal
  projection of a data point into an element.}{Schematic of an orthogonal
  projection of a data point into an element. A data point at the geometric
  location $\dpt{\vect{z}}{d}$ is projected into an element,
  $\dpt{\varepsilon}{d}$, at a parametric element coordinate, $\dpt{\xi}{d}$,
  and geometric position $\fnof{\vect{z}}{\dpt{\xi}{d}}$ such that the
  distance between the projection and the data point is
  minimised.}{fig:dataprojection}

The measure of the error for each data point, $\dpt{f}{d}$, is defined as the
Euclidean distance between the data point and its closest projection onto the
current mesh \ie
\begin{equation}
  \fnof{\dpt{f}{d}}{\dpt{\vect{\xi}}{d}}=\norm{\fnof{\vect{z}}{
      \dpt{\vect{\xi}}{d}}-\dpt{\vect{z}}{d}}
\end{equation}

For a given projection of the data points onto the mesh (\ie if
$\dpt{\vect{\xi}}{d}$ are held constant), the \emph{data objective} function
to be minimised in the fit is formed from the weighted sum of squares of the
individual errors \ie
\begin{equation}
  \fnof{\mathcal{F}}{\vect{u}}=\gsum{e=1}{E}{\gsum{d=1}{D_{e}}{\dpt{\gamma}{d}
      \pbrac{\fnof{\dpt{f}{d}}{\dpt{\vect{\xi}}{d}}}^{2}}}=\gsum{e=1}{E}{
    \gsum{d=1}{D_{e}}{\dpt{\gamma}{d}\norm{\fnof{\vect{z}}{\dpt{
            \vect{\xi}}{d}}-\dpt{\vect{z}}{d}}^{2}}}
  \label{eqn:ldataresidualfn}
\end{equation}
where $\dpt{\gamma}{d}$ is a weight for each data point, $D_{e}$ the number of
data points within an element $e$ and $\vect{u}$ is a vector of mesh
parameters (nodal positions, derivatives and scale factors).

The fitting problem is hence to find the set of mesh parameters that minimises
this objective function. Substituting \eqnref{eqn:dataprojectionpos} into
\eqnref{eqn:ldataresidualfn} and differentiating with respect to the mesh
parameters within an element $e$ yields
\begin{equation}
  \begin{split}
    \delby{\fnof{\mathcal{F}}{\vect{u}}}{\npt{x}{\beta}_{j,v}}&=2
    \gsum{d=1}{D_{e}}{\dpt{\gamma}{d}\pbrac{\gbfn{\alpha}{u}{
          \dpt{\vect{\xi}}{d}}\npt{x}{\alpha}_{j,u}\gsf{(\alpha)}{(u)}-
        \dpt{z}{d}_{j}}\gbfn{\beta}{v}{\dpt{\vect{\xi}}{d}}
      \gsf{(\beta)}{(v)}} \\
    \delby{\fnof{\mathcal{F}}{\vect{u}}}{\gsf{\beta}{v}}&=2\gsum{d=1}{D_{e}}{
      \dpt{\gamma}{d}\dotprod{\pbrac{\gbfn{\alpha}{u}{\dpt{\vect{\xi}}{d}}
          \npt{\vect{x}}{\alpha}_{,u}\gsf{\pbrac{\alpha}}{\pbrac{u}}-
          \dpt{\vect{z}}{d}}}{\gbfn{\beta}{v}{\dpt{\vect{\xi}}{d}}
        \npt{\vect{x}}{\beta}_{,v}}}
  \end{split}
  \label{eqn:derivldataresidualfn}
\end{equation}

A minimum of the objective function is hence found by setting the derivatives
of the objective function in \eqnref{eqn:derivldataresidualfn} to zero.  This
will only result in a linear system if the scale factors are kept constant
during the fit. That is, the vector $\vect{u}$ will only contain the nodal
positions and the nodal \arclen derivatives.  With this restriction the
derivative of \eqnref{eqn:ldataresidualfn} with respect to the scale factors
can be ignored and the minimum of the objective function is given by
\begin{equation}
  \gsum{d=1}{D_{e}}{\dpt{\gamma}{d}\gbfn{\alpha}{u}{\dpt{\vect{\xi}}{d}}
    \gsf{(\alpha)}{(u)}\gbfn{\beta}{v}{\dpt{\vect{\xi}}{d}}\gsf{(\beta)}{(v)}}
  \npt{x}{\alpha}_{j,u}=\gsum{d=1}{D_{e}}{\dpt{\gamma}{d}\gbfn{\beta}{v}
    {\dpt{\vect{\xi}}{d}}\gsf{(\beta)}{(v)}\dpt{z}{d}_{j}}
\end{equation}
that is, a linear elemental finite element stiffness matrix and residual
vector system for each geometric index $j$ of the form
\begin{equation}
  K_{\alpha v}^{\beta u}\npt{u}{\alpha}_{j,u}=\npt{r}{\beta}_{vj}
  \label{eqn:fittingelemstiffnessystem}
\end{equation}
where the scope of the $\alpha$, $\beta$, $u$ and $v$ indices depend on the
basis functions used to interpolate the \nth{j} geometric variable.

A linear system of equations governing the entire mesh is found by assembling
a global stiffness matrix from all of the individual element matrices. This is
then solved to yield the nodal positions and derivatives that minimise the
error in the mesh.

As the scale factors and data point projections are held constant during the
above fitting procedure the resultant mesh may not completely minimise the
error. It is hence necessary to iteratively apply the fitting procedure in
order to obtain a mesh that best fits the data. The fit can be considered
converged when the RMS error in the fit does not change significantly
between iterations. The RMS error in the fit is defined as
\begin{equation}
  \text{RMS error}=\sqrt{\dfrac{\gsum{d=1}{D}{\norm{\fnof{\vect{z}}{
            \dpt{\vect{\xi}}{d}}-\dpt{\vect{z}}{d}}^{2}}}{D}}
\end{equation}

The linear fitting algorithm used in this thesis is hence:
\begin{enumerate}
\item Define an initial mesh (and calculate the initial nodal scale factors).
  \item Calculate the initial data point projections and initial error in the
    mesh.
  \item Repeat until converged or the maximum number of iterations is exceeded.
  \begin{enumerate}
  \item Fit the mesh by applying the linear fitting procedure.
  \item Update the scale factors to be average \arclens based on the new
    mesh.
  \item Recalculate the data point projections on the new mesh.
  \end{enumerate}
\end{enumerate}

\subsubsection{$\xi$ point calculation}
\label{sec:xipointcalculation}

The fitting method described above requires the location of the orthogonal
projection of a data point (the point at which the distance between the data
point and the projection is a minimum) onto the mesh to be determined. To find
this point consider the distance between a data point and it's projection,
$\dpt{f}{d}$, as the Euclidean norm of the error vector from some point (at a
parametric elemental position of $\vect{\xi}$), $\fnof{\vect{z}}{\vect{\xi}}$,
and the location of the data point, $\dpt{\vect{z}}{d}$ \ie
\begin{equation}
  \begin{split}
    \fnof{\dpt{f}{d}}{\vect{\xi}}&=\norm{\fnof{\dpt{\vect{e}}{d}}{\vect{\xi}}} \\ 
    &=\norm{\fnof{\vect{z}}{\vect{\xi}}-\dpt{\vect{z}}{d}} \\ 
    &=\norm{\gbfn{\alpha}{u}{\vect{\xi}}\npt{\vect{x}}{\alpha}_{,u}
      \gsf{\pbrac{\alpha}}{\pbrac{u}}-\dpt{\vect{z}}{d}}
  \end{split}
\end{equation}

The parametric elemental location which minimises this distance (assuming
that it exists within the element), $\dpt{\vect{\xi}}{d}$, is hence given by
\begin{equation}
  \delby{\fnof{\dpt{f}{d}}{\dpt{\vect{\xi}}{d}}}{\xi_{l}}=\dfrac{\delby{\gbfn{
        \alpha}{u}{\dpt{\vect{\xi}}{d}}}{\xi_{l}}\dotprod{\npt{\vect{x}}{
        \alpha}_{,u}\gsf{\pbrac{\alpha}}{\pbrac{u}}}{\fnof{\dpt{\vect{e}}{d}}{\dpt{
          \vect{\xi}}{d}}}}{\norm{\fnof{\dpt{\vect{e}}{d}}{\dpt{\vect{\xi}}{d}}}}=0
\end{equation}
with $l$ varying from $1$ to the number of $\xi$ coordinates in the mesh under
consideration.

The parametric projection location, $\dpt{\vect{\xi}}{d}$, can hence be found
from the (vector) root of the (vector) function
\begin{equation}
  \fnof{g_{l}}{\vect{\xi}}=\dotprod{\delby{\fnof{\vect{z}}{\vect{\xi}}}{
      \xi_{l}}}{\fnof{\dpt{\vect{e}}{d}}{\vect{\xi}}}=0
\end{equation}

This root is easily found using an iterative Newton-Raphson method. If an
initial location, $\vect{\xi}$, is assumed a correction to this position,
$\Delta\vect{\xi}$, can be found from 
\begin{equation}
  \Delta\vect{\xi}=-\inverse{\fnof{\matr{J}}{\fnof{\vect{g}}{\vect{\xi}}}}
  \fnof{\vect{g}}{\vect{\xi}}
\end{equation}
where $\fnof{\matr{J}}{\fnof{\vect{g}}{\vect{\xi}}}$ is the \emph{Jacobian
  matrix} given by
\begin{equation}
  \fnof{\matr{J}}{\fnof{\vect{g}}{\vect{\xi}}}=\begin{bmatrix}
    \delby{g_{1}}{\xi_{1}} & \delby{g_{1}}{\xi_{2}} & \ldots & 
    \delby{g_{1}}{\xi_{n}} \\
    \delby{g_{2}}{\xi_{1}} & \delby{g_{2}}{\xi_{2}} & \ldots & 
    \delby{g_{2}}{\xi_{n}} \\
    \vdots & \vdots & \ddots & \vdots \\
    \delby{g_{m}}{\xi_{1}} & \delby{g_{m}}{\xi_{2}} & \ldots & 
    \delby{g_{m}}{\xi_{n}}
  \end{bmatrix}
\end{equation}

For the function considered here
\begin{equation}
  \fnof{J_{ij}}{\fnof{\vect{g}}{\vect{\xi}}}=\dotprod{\deltwoby{\fnof{
        \vect{z}}{\vect{\xi}}}{\xi_{i}}{\xi_{j}}}{\fnof{\dpt{\vect{e}}{d}}{
      \vect{\xi}}}+\dotprod{\delby{\fnof{\vect{z}}{\vect{\xi}}}{\xi_{i}}}{
    \delby{\fnof{\dpt{\vect{e}}{d}}{\vect{\xi}}}{\xi_{j}}}
\end{equation}

The algorithm to find the orthogonal projections is hence:
\begin{enumerate}
\item For each data point, d.
  \begin{enumerate}
  \item For each element.
    \begin{enumerate}
    \item Guess an initial starting location for the projection (\eg the
      middle of the element).
    \item Iteratively apply the Newton-Raphson method until (a) The projection
      is converged; (b) The maximum number of Newton-Raphson iterations is
      exceeded; (c) The projection moves outside the element.
    \item If a projection has been found inside the element and the distance
      from this projection to the data point is smaller than other projections
      in other elements set $\dpt{\varepsilon}{d}$ to this element and
      $\dpt{\vect{\xi}}{d}$ to this projection.
    \end{enumerate}
  \end{enumerate}
\end{enumerate}

The initial calculation of the orthogonal projections is computationally
expensive although the method is parallelisable at the data point loop level
and steps can be taken to ensure that excessive
computation is avoided (\eg immediately disregard elements at early stages in
the Newton-Raphson iteration whose projections yield distances far greater
than the current smallest projection). Once the initial projections have been
calculated then, if the mesh is changed (as part of an iterative fitting
procedure), the projections can be quickly and easily updated by applying the
Newton-Raphson method from the current projections.

\subsubsection{Sobelov smoothing for linear fitting}
\label{sec:sobelovlinearfitting}

With an insufficient number of data points, fitting 'noisy' data or fitting
data that has an uneven spread, a smoothness constraint \cite{young:1989}
can be introduced by adding a second term to the objective function:
\begin{equation}
  \fnof{F }{\vect{u}}=\gsum{e=1}{E}{\gsum{d=1}{D_{e}}{\dpt{\gamma}{d}
    \norm{\fnof{\vect{z}}{\dpt{\vect{\xi}}{d}}-\dpt{\vect{z}}{d}}^{2}}}+
  \gint{\Omega}{}{\fnof{g}{\fnof{\vect{u}}{\vect{\xi}}}}{\Omega}
  \label{eqn:sobeloverrorfn}
\end{equation}

The first term in \eqnref{eqn:sobeloverrorfn} measures the error in the
surface from the data and the second term measures deformation of the surface.
To measure the deformation of the surface a $\nth{p}$ order weighted
Sobelov norm \cite{terzopoulos:1986,zzz-tikhonov:1977} is used, defined by
\begin{equation}
  \fnof{g_{p,w}}{\fnof{\vect{u}}{\vect{\xi}}}=\gsum{q=0}{p}{\gsum{i+j=q}{}
  {w_{ij}\norm{\delntwoby{q}{\vect{u}}{\xione}{i}{\xitwo}{j}}^{2}}}
  \label{eqn:Sobelovnorm}
\end{equation}
where $w_{ij}$ are the weights for the norm. The addition to the objective
function, called the \emph{Sobelov value}, is defined as
\begin{equation}
  \fnof{G}{\fnof{\vect{u}}{\vect{\xi}}}=\gint{\Omega}{}{\fnof{g}{
      \fnof{\vect{u}}{\vect{\xi}}}}{\Omega}
  \label{eqn:Sobelovvalue}
\end{equation}
where $\Omega$ is the mesh domain.

For the case of \twodal fitting (\eg \cubicherm elements) $j=0$ (as there is
no $\xitwo$ direction) and for the case of \threedal fitting (\eg \bicubicherm
elements) $j=0,..,2$. Consider the case for $p=2$, and for \twods: $w_{0}=0,
w_{1}=\tau, w_{2}=\kappa$, and for \threeds: $w_{00}=0, w_{01}=w_{10}=\tau,
w_{20}=w_{02}= \kappa, w_{11}=2\kappa$. In \twods the Sobelov value now becomes
\begin{equation}
  \fnof{G}{\fnof{\vect{u}}{\vect{\xi}}}=\goneint{\pbrac{\tau
      \norm{\dby{\vect{u}}{\xi}}^{2}+\kappa\norm{\dtwosqby{
          \vect{u}}{\xi}}^{2}}}{\Omega}
  \label{eqn:2DptwoSobelovnorm}
\end{equation}
and in \threeds it becomes
\begin{equation}
  \fnof{G}{\fnof{\vect{u}}{\vect{\xi}}}=\goneint{\pbrac{\tau\pbrac{
        \norm{\delby{\vect{u}}{\xione}}^{2}+\norm{\delby{\vect{u}}
          {\xitwo}}^{2}}+\kappa\pbrac{\norm{\deltwosqby{\vect{u}}{\xione}}^{2}+
        2\norm{\deltwoby{\vect{u}}{\xione}{\xitwo}}^{2}+
        \norm{\deltwosqby{\vect{u}}{\xitwo}}^{2}}}}{\Omega}
  \label{eqn:3DptwoSobelovnorm}
\end{equation}

Thus the parameter $\tau$ controls the tension of the surface and the
parameter $\kappa$ controls the degree of surface curvature
\cite{terzopoulos:1986}.

The additional elemental stiffness matrix terms for
\eqnref{eqn:fittingelemstiffnessystem} are
\begin{equation}
  K_{\alpha v}^{\beta u}=2\gsf{\pbrac{\alpha}}{\pbrac{u}}
  \gsf{\pbrac{\beta}}{\pbrac{v}}\gint{0}{1}{\pbrac{\tau\delby{
        \gbfn{\alpha}{u}{\xi}}{\xi}\delby{\gbfn{\beta}{v}{\xi}}{\xi}+
      \kappa\deltwosqby{\gbfn{\alpha}{u}{\xi}}{\xi}
      \deltwosqby{\gbfn{\beta}{v}{\xi}}{\xi}}\fnof{J}{\xi}}{\xi}
\end{equation}
for \twods and 
\begin{multline}
  K_{\alpha v}^{\beta u}=2\gsf{\pbrac{\alpha}}{\pbrac{u}}
  \gsf{\pbrac{\beta}}{\pbrac{v}}\dintl{\vect{0}}{\vect{1}}\left(\tau\pbrac{
      \delby{\gbfn{\alpha}{u}{\vect{\xi}}}{\xione}
      \delby{\gbfn{\beta}{v}{\vect{\xi}}}{\xione}+
      \delby{\gbfn{\alpha}{u}{\vect{\xi}}}{\xitwo}
      \delby{\gbfn{\beta}{v}{\vect{\xi}}}{\xitwo}}+\right. \\
  \left.\kappa\pbrac{\deltwosqby{\gbfn{\alpha}{u}{\vect{\xi}}}{\xione}
      \deltwosqby{\gbfn{\beta}{v}{\vect{\xi}}}{\xione}+
      \deltwoby{\gbfn{\alpha}{u}{\vect{\xi}}}{\xione}{\xitwo}
      \deltwoby{\gbfn{\beta}{v}{\vect{\xi}}}{\xione}{\xitwo}+
      \deltwosqby{\gbfn{\alpha}{u}{\vect{\xi}}}{\xitwo}
      \deltwosqby{\gbfn{\beta}{v}{\vect{\xi}}}{\xitwo}}\right)
  \jacobian\,d\vect{\xi}
\end{multline}
for \threeds.

\subsection{Non-linear Fitting}
\label{sec:nonlinearfitting}

\subsubsection{Problem formulation}
\label{sec:problemformulation}

One problem that arises when using linear fitting with \cubicherm elements is
that \arclen derivatives and average \arclen scaling are not maintained during
the fit. In linear fitting there is no information supplied in the linear
fitting model that enforces \arclen derivatives and the scale factors are held
constant during the fit. In this section the question of how to fit the data
whilst maintaining \arclen derivatives and an even spacing of \arclen with
$\xi$ is considered.  Because both the value of the \arclen for the element
and the relationship between the derivatives in the various spatial directions
depend upon the mesh parameters in a \nonlin fashion, the only way to ensure
that \arclen derivatives are maintained during fitting is to use a \nonlin
fitting procedure.

Consider the following: Let $\vect{u}$ be the vector of mesh parameters,
$\dpt{\vect{z}}{d}$ be the vector of the location of data point $d$ in space
and $\fnof{\vect{z}}{\vect{u},\dpt{\vect{\xi}}{d}}$ the vector of the location
of the closest projection (at the points given by the vector
$\dpt{\vect{\xi}}{d}$) of that data point onto the mesh. Now the error in the
$d^{\text{th}}$ data point can be expressed in terms of an error vector
\begin{equation}
  \fnof{\dpt{\vect{e}}{d}}{\vect{u},\dpt{\vect{\xi}}{d}}=\fnof{\vect{z}}
  {\vect{u},\dpt{\vect{\xi}}{d}}-\dpt{\vect{z}}{d}
  \label{eqn:errorvector}
\end{equation}
and a residual
\begin{equation}
  \fnof{\dpt{f}{d}}{\vect{u},\dpt{\vect{\xi}}{d}}=\dpt{\gamma}{d}\norm{\fnof{
      \dpt{\vect{e}}{d}}{\vect{u},\dpt{\vect{\xi}}{d}}}^{2}
  \label{eqn:dataresidual}
\end{equation}

A \emph{data objective} function,
$\fnof{\mathcal{F}}{\vect{u},\dpt{\vect{\xi}}{d}}$, is formed as the sum of
squares of the individual residuals for the mesh $\vect{u}$ and projections
$\dpt{\vect{\xi}}{d}$
\begin{equation}
  \fnof{\mathcal{F}}{\vect{u},\dpt{\vect{\xi}}{d}}=\fntof{\vect{f}}{\vect{u},
    \dpt{\vect{\xi}}{d}}\fnof{\vect{f}}{\vect{u},\dpt{\vect{\xi}}{d}}
  \label{eqn:nldataresidualfn}
\end{equation}
The fitting problem then becomes, for constant $\dpt{\vect{\xi}}{d}$,
\begin{equation}
  \min_{\vect{u} \in \mathbb{R}} \quad \fnof{\mathcal{F}}{\vect{u}}=\fntof{\vect{f}}
  {\vect{u},\dpt{\vect{\xi}}{d}}\fnof{\vect{f}}{\vect{u},\dpt{\vect{\xi}}{d}}
\end{equation}

In order to maintain \arclen derivatives a \nonlin constraint is needed.  This
constraint comes from the geometric properties of \arclen derivatives.  For a
global node $A$ and $\xi$-direction $l$ the magnitude of the vector of \arclen
derivatives in the various spatial directions must be of unit length as
given in \eqnref{eqn:chnormconstraint}.  Hence the constraint is
\begin{equation}
  \npt{c}{A}_{l}=\norm{\npt{\vect{x}}{A}_{,l}}=1
  \label{eqn:normconstraint}
\end{equation}
for $l=1,2$.

This also implies simple bounds on the derivative variables:
\begin{equation}
  -1 \leq \npt{x}{A}_{j,l} \leq 1
\end{equation}

The fitting problem can thus be written in terms of the non-linearly
constrained, \nonlin optimisation problem
\begin{equation}
  \begin{split}
    \min_{\vect{u}\in\mathbb{R},~\dpt{\vect{\xi}}{d}\text{ constant}}\quad &
    \fnof{\mathcal{F}}{\vect{u}}=\fntof{\vect{f}}{\vect{u},
      \dpt{\vect{\xi}}{d}}\fnof{\vect{f}}{\vect{u},\dpt{\vect{\xi}}{d}} \\
    \text{subject to} \quad & \vect{a} \leq 
    \begin{pmatrix} 
      \vect{u} \\ 
      \fnof{\vect{c}}{\vect{u}}
    \end{pmatrix} \leq \vect{b}
  \end{split}
  \label{eqn:nloproblem}
\end{equation}
where $\vect{a}$ is a vector of lower bounds, $\vect{b}$ a vector of upper
bounds and $\fnof{\vect{c}}{\vect{u}}$ a vector of \nonlin constraints. This
type of optimisation problem can be solved with readily available \nonlin
optimisation packages such as NPSOL \cite{gill:1986}.

In order to ensure that there is an approximately uniform spacing of $\xi$
with \arclen two approaches can be used. The first approach is to use an
iterative technique for the scale factors (as per the linear fitting case) and
is detailed here. The second approach is to include the scale factors in the
optimisation problem and is detailed in the next sub-section.

The constraint on the nodal scale factors (being the average of the line
lengths either side of the node) is placed upon the problem to ensure that the
\arclen to $\xi$ spacing is approximately uniform. As this only yields
approximately uniform \arclen to $\xi$ spacing the constraint on the nodal
scale factors can be relaxed \ie if the nodal scale factors are held fixed
during the fit then although there will not be average \arclen scaling
throughout the fitting process there will be a reasonable approximation to it.
With the nodal scale factors fixed the variables $\gsf{A}{l}$ can be removed
from the vector of mesh parameters $\vect{u}$.

Hence, the \nonlin fitting approach is to hold the scale factors constant, fit
the mesh with these scale factors, and then update the scale factors (based on
the new mesh) to be average \arclen. This process can be repeated iteratively
until the desired fit has been achieved. It should be noted that as
\eqnref{eqn:nloproblem} is only defined for constant data point projections,
this iterative approach means that the data point projections can also be
updated at the same time as the scale factors.

The convergence of the fitting problem can be measured in two ways. The first
is the convergence of the RMS error in the data and second is the convergence
in the magnitude of the nodal scale factors.

The algorithm is therefore:
\begin{enumerate}
\item Define an initial mesh (and calculate the initial nodal scale factors).
  \item Calculate the initial data point projections and initial error in the
    mesh.
  \item Repeat until converged or the maximum number of iterations is exceeded.
  \begin{enumerate}
    \item Fit the mesh to the data by solving \eqnref{eqn:nloproblem}.
    \item Update the scale factors to be average \arclens based on the new
      mesh.
    \item Recalculate the data point projections on the new mesh.
  \end{enumerate}
\end{enumerate}

\subsubsection{Alternative formulation}
\label{sec:alternativeformulation}

An alternative approach to ensuring average \arclen scale factors can be
formulated by including the scale factors in the optimisation problem as
optimisation variables.  The vector of mesh parameters, $\vect{u}$, is hence
extended to include the nodal scale factors. With this a new constraint is
introduced to \eqnref{eqn:nloproblem} to ensure that a nodal scale factor is
the average of the \arclens on either side of that node. If $\gsf{A}{l}$ is
the nodal scale factor for global node $A$ in the $s_{l}$ direction, as given
by \eqnref{eqn:avearclenscale}, then the constraint that the nodal scale
factor equals the average \arclen is given by
\begin{equation}
  \npt{c}{A}_{l}=\dfrac{\fnof{s_{l}}{\fnof{A_{\ominus}}{l}}+
    \fnof{s_{l}}{\fnof{A_{\oplus}}{l}}}{2}-\gsf{A}{l}=0
\end{equation}
or
\begin{equation} 
  \npt{c}{A}_{l}=\dfrac{1}{2}\pbrac{
    \gint{0}{1}{\norm{\delby{\fnof{\vect{x}}{\vect{\xi}}}{\xi_{l}}}}{
      \fnof{\xi_{l}}{\fnof{A_{\ominus}}{l}}}+
    \gint{0}{1}{\norm{\delby{\fnof{\vect{x}}{\vect{\xi}}}{\xi_{l}}}}{
      \fnof{\xi_{l}}{\fnof{A_{\oplus}}{l}}}}-\gsf{A}{l}=0
  \label{eqn:avarclenconstraint}
\end{equation}
where $\fnof{A_{\ominus}}{l}$ is the element immediately preceding (in the
\nth{l} direction) node $A$, and $\fnof{A_{\oplus}}{l}$ is the element
immediately after (in the \nth{l} direction) node $A$. Note that when the
infinitesimal in the integrals above contain
$\fnof{A_{\ominus}}{l}$ ($\fnof{A_{\oplus}}{l}$) the $\xi_{l}$ variable is
assumed to be taken over the previous (next) element from the global node $A$
and thus the integrals are always \onedal.

This also generates a simple bound on the scale-factors:
\begin{equation}
  \gsf{A}{l} > 0
\end{equation}

This alternative formulation of the \nonlin fitting problem does have one
practical limitation. With average \arclen scaling the interpolation within an
element depends on the \arclen in the neighbouring elements, and the \arclen
in the neighbouring elements depends on their neighbouring elements and so on.
This results in a \emph{global mesh} in that every part of the mesh is
dependent on every other part of the mesh. The implication of this is that the
entire mesh must be included in the fit otherwise average \arclen scaling
cannot be achieved. This is clearly not a desirable feature for very large
problems or for problems where only a small part of the mesh is in error and
needs to be fitted. 

This formulation still requires iteration to converge the data point
projections but does have the advantage that the number of iterations required
is reduced as the scale factors are found during the fit. This is at the
expense of having to solve a much larger \nonlin optimisation problem with
more variables and, more importantly, more \nonlin constraints.

The fit can be considered converged when the RMS error in the fit does not
change significantly between iterations.  The algorithm for the \nonlin data
fitting procedure is as follows:
\begin{enumerate}
\item Define an initial mesh (and calculate the initial nodal scale factors).
\item Calculate the initial data point projections.
\item Repeat until converged or the maximum number of iterations is exceeded.
  \begin{enumerate}
  \item Fit the mesh to the data by solving \eqnref{eqn:nloproblem} (with
    the new constraints).
  \item Recalculate the data point projections on the new mesh.
  \end{enumerate}
\end{enumerate}

\subsubsection{Residual and constraint Jacobians}
\label{sec:resandcontjacs}

Solution of the \nonlin problem given in \eqnref{eqn:nloproblem} will
generally require the evaluation of the objective gradient (or residual vector
Jacobian) and the constraint Jacobian with respect to the optimisation (mesh)
parameters. These optimisation parameters fall into two basic types of
variable: the nodal variables, $\npt{x}{\beta}_{j,v}$, and the nodal scale
factors, $\gsf{\beta}{v}$.

To find the residual Jacobian consider the error vector for the data point
$d$, $\dpt{\vect{e}}{d}$, defined in \eqnref{eqn:errorvector}, and its
corresponding residual, $\dpt{f}{d}$, defined in \eqnref{eqn:dataresidual}. The
position of the projection of data point $d$ within the element
$\dpt{\varepsilon}{d}$ is given by \eqnref{eqn:dataprojectionpos}. Hence the
Jacobian of the residual vector for the various optimisation variables within
$\dpt{\varepsilon}{d}$ can be found by substituting
\eqnref{eqn:dataprojectionpos} into
\eqnrefs{eqn:errorvector}{eqn:dataresidual} and differentiating \ie
\begin{align}
  \delby{\dpt{f}{d}}{\npt{x}{\beta}_{j,v}}&=2\gbfn{\beta}{v}{
    \dpt{\vect{\xi}}{d}}\gsf{\pbrac{\beta}}{\pbrac{v}}\dpt{e}{d}_{j} 
  \label{eqn:dataresjacnodevars} \\
  \delby{\dpt{f}{d}}{\gsf{\beta}{v}}&=2\gbfn{\beta}{v}{\dpt{\vect{\xi}}{d}}
  \dotprod{\npt{\vect{x}}{\pbrac{\beta}}_{,\pbrac{v}}}{\dpt{\vect{e}}{d}}
  \label{eqn:dataresjacsfvars}
\end{align}
For the optimisation variables not contained within $\dpt{\varepsilon}{d}$ the
Jacobian is zero.
  
Note that if the data residual is defined as
\begin{equation}
  \fnof{\dpt{f}{d}}{\vect{u},\dpt{\vect{\xi}}{d}}=\norm{
    \fnof{\dpt{\vect{e}}{d}}{\vect{u},\dpt{\vect{\xi}}{d}}}
\end{equation}
then \eqnref{eqn:dataresjacnodevars} becomes
\begin{equation}
  \delby{\dpt{f}{d}}{\npt{x}{\beta}_{j,v}}=\dfrac{\gbfn{\beta}{v}{
      \dpt{\vect{\xi}}{d}}\gsf{\pbrac{\beta}}{\pbrac{v}}
    \dpt{e}{d}_{j}}{\dpt{f}{d}}
\end{equation}
which is singular at the optimal solution $\dpt{f}{d}=0$. To avoid numerical
problems the residual is therefore defined by \eqnref{eqn:dataresidual} as the
square of the Euclidean distance rather than the straight Euclidean distance
as per the linear fitting case.

Now consider the constraint Jacobians. Differentiating the first constraint,
\eqnref{eqn:normconstraint}, with respect to the global mesh parameters gives
the constraint gradients:
\begin{align}
  \delby{\npt{c}{A}_{l}}{\npt{x}{B}_{j,v}}&=\kronecker{A}{B}
  \kronecker{l}{v}\dfrac{\npt{x}{B}_{j,v}}{\norm{
      \npt{\vect{x}}{B}_{,v}}} \label{eqn:normjacderiv} \\ 
  \delby{\npt{c}{A}_{l}}{\gsf{B}{v}}&=0 \label{eqn:normjacline}
\end{align}
for $l=1,2$.

To calculate the gradients for the second constraint,
\eqnref{eqn:avarclenconstraint}, the rate of change of $\vect{x}$ with respect
to $\xi_{l}$ within an element is first computed by differentiating the
generalised form of \eqnref{eqn:chsfinterpolation} \ie
\begin{equation}
  \delby{\fnof{\vect{x}}{\vect{\xi}}}{\xi_{l}}=\delby{\gbfn{\alpha}{u}{
      \vect{\xi}}}{\xi_{l}}\npt{\vect{x}}{\alpha}_{,u}\gsf{\pbrac{
      \alpha}}{\pbrac{u}}
  \label{eqn:delxdelxiinterp}
\end{equation}

Substituting \eqnref{eqn:delxdelxiinterp} into \eqnref{eqn:avarclenconstraint}
and differentiating with respect to the optimisation variables gives the
constraint gradients: 
\begin{multline}
  \delby{c_{l}^{A}}{\npt{x}{B}_{j,v}}=\dfrac{1}{2}\left(\kronecker{B}{
      \fnof{\Delta}{\beta,\fnof{A_{\ominus}}{l}}}\gint{0}{1}{\dfrac{\delby{
          \gbfn{\beta}{v}{\vect{\xi}}}{\xi_{l}}\gsf{\pbrac{\beta}}{\pbrac{v}}
        \delby{\fnof{x_{j}}{\vect{\xi}}}{\xi_{l}}}{\norm{\delby{\fnof{
              \vect{x}}{\vect{\xi}}}{\xi_{l}}}}}{\fnof{\xi_{l}}{
        \fnof{A_{\ominus}}{l}}+\right.\\  
    \left.\kronecker{B}{\fnof{\Delta}{\beta,\fnof{A_{\oplus}}{l}}}\gint{0}{1}{
        \dfrac{\delby{\gbfn{\beta}{v}{\vect{\xi}}}{\xi_{l}}
          \gsf{\pbrac{\beta}}{\pbrac{v}}\delby{\fnof{x_{j}}{
              \vect{\xi}}}{\xi_{l}}}{\norm{\delby{\fnof{\vect{x}}{
                \vect{\xi}}}{\xi_{l}}}}}{\fnof{\xi_{l}}{
          \fnof{A_{\oplus}}{l}}}}\right) 
  \label{eqn:arclenjacnode}
\end{multline}
and
\begin{multline}  
  \delby{c_{l}^{A}}{\gsf{B}{v}}=\dfrac{1}{2}\left(\kronecker{B}{
      \fnof{\Delta}{\beta,\fnof{A_{\ominus}}{l}}}\gint{0}{1}{\dfrac{\delby{
          \gbfn{\beta}{v}{\vect{\xi}}}{\xi_{l}}\dotprod{\npt{
            \vect{x}}{\beta}_{,v}}{\delby{\fnof{\vect{x}}{\vect{\xi}}}{
            \xi_{l}}}}{\norm{\delby{\fnof{\vect{x}}{\vect{\xi}}}{\xi_{l}}}}}{
      \fnof{\xi_{l}}{\fnof{A_{\ominus}}{l}}}+\right. \\ 
  \left.\kronecker{B}{\fnof{\Delta}{\beta,\fnof{A_{\oplus}}{l}}}\gint{0}{1}{
      \dfrac{\delby{\gbfn{\beta}{v}{\vect{\xi}}}{\xi_{l}}\dotprod{\npt{\vect{x}}{
            \beta}_{,v}}{\delby{\fnof{\vect{x}}{\vect{\xi}}}
          {\xi_{l}}}}{\norm{\delby{\fnof{\vect{x}}{\vect{\xi}}}{\xi_{l}}}}}{
      \fnof{\xi_{l}}{\fnof{A_{\oplus}}{l}}}\right)-
  \kronecker{A}{B}\kronecker{l}{v} \label{eqn:arclenjacline}
\end{multline}
where $\beta$ ranges from one to the number of local nodes in the element.
Again when the infinitesimal in the integrals above contain
$\fnof{A_{\ominus}}{l}$ ($\fnof{A_{\oplus}}{l}$) the $\xi_{l}$ variable is
assumed to be taken over the previous (next) element and the integrals are
thus always \onedal. In the above formulae the Kronecker delta is used to
extract out the local node and derivative that corresponds to the global
degree-of-freedom which the gradient is taken with respect to.

\subsubsection{Sobelov smoothing for non-linear fitting}

To implement Sobelov smoothing in the form of the \nonlin optimisation problem
given in \eqnref{eqn:nloproblem} the residual vector is modified by defining
an additional residual equal to the Sobelov value for the mesh being fitted \ie
\begin{equation}
  \fnof{f^{D+1}}{\vect{u},\vect{\xi}}=\fnof{G}{\fnof{\vect{u}}{\vect{\xi}}}
  \label{eqn:addSobresidual}
\end{equation}
where $\fnof{G}{\fnof{\vect{u}}{\vect{\xi}}}$ is defined in
\eqnref{eqn:Sobelovvalue}.  The objective function to minimise now becomes
\begin{equation}
  \fnof{F}{\vect{u}}=\fnof{\mathcal{F}}{\vect{u}}+\fnof{G^{2}}{\vect{u}}=
  \fnof{\mathcal{F}}{\vect{u}}+\fnof{\mathcal{G}}{\vect{u}}
\end{equation}
where $\fnof{\mathcal{F}}{\vect{u}}$ is the data objective defined in
\eqnref{eqn:nldataresidualfn} and
$\fnof{\mathcal{G}}{\vect{u}}=\fnof{G^{2}}{\vect{u}}$ is the \emph{Sobelov
objective}.

The gradients of this additional residual can then be found by differentiating
either \eqnref{eqn:2DptwoSobelovnorm} or \eqnref{eqn:3DptwoSobelovnorm} with
respect to the optimisation (mesh) parameters. For \twodal fitting these
gradients can be found by breaking the domain up into elements \ie
\begin{equation}
  \fnof{G}{\vect{u}}=\dsum_{e}\gint{0}{1}{\pbrac{\tau\norm{\delby{
          \fnof{\vect{x}}{\xi}}{\xi}}^{2}+\kappa\norm{\deltwosqby{\fnof{
            \vect{x}}{\xi}}{\xi}}^{2}}\fnof{J}{\xi}}{\xi}
\end{equation}

Interpolating within an element and differentiating with respect to the
different optimisation parameters within that element gives the additional
residual gradients. For \twods these are
\begin{align}
  \delby{f^{D+1}}{\npt{x}{\beta}_{j,v}}&=2\gint{0}{1}{\pbrac{\tau
      \delby{\gbfn{\beta}{v}{\xi}}{\xi}\gsf{\pbrac{\beta}}{\pbrac{v}}
      \delby{\fnof{x_{j}}{\xi}}{\xi}+\kappa\deltwosqby{\gbfn{\beta}{v}{\xi}}{\xi}
      \gsf{\pbrac{\beta}}{\pbrac{v}}
      \deltwosqby{\fnof{x_{j}}{\xi}}{\xi}}\fnof{J}{\xi}}{\xi} 
  \label{eqn:2Dsobjacnode} \\
  \delby{f^{D+1}}{\gsf{\beta}{v}}&=2\gint{0}{1}{\pbrac{\tau\delby{
        \gbfn{\beta}{v}{\xi}}{\xi}\dotprod{\npt{\vect{x}}{\beta}_{,v}}{
        \delby{\fnof{\vect{x}}{\xi}}{\xi}}+\kappa\deltwosqby{
        \gbfn{\beta}{v}{\xi}}{\xi}\dotprod{\npt{\vect{x}}{\beta}_{,v}}{
        \deltwosqby{\fnof{\vect{x}}{\xi}}{\xi}}}\fnof{J}{\xi}}{\xi}
  \label{eqn:2Dsobjacline}
\end{align}

In \threeds the additional residual gradients become
\begin{multline}
  \delby{f^{D+1}}{\npt{x}{\beta}_{j,v}}=2\dintl{\vect{0}}{\vect{1}}\left(\tau
    \pbrac{\delby{\gbfn{\beta}{v}{\vect{\xi}}}{\xione}\gsf{\pbrac{\beta}}{
        \pbrac{v}}\delby{\fnof{x_{j}}{\vect{\xi}}}{\xione}+\delby{
        \gbfn{\beta}{v}{\vect{\xi}}}{\xitwo}\gsf{\pbrac{\beta}}{\pbrac{v}}
      \delby{\fnof{x_{j}}{\vect{\xi}}}{\xitwo}}+\right. \\
  \left.\kappa\pbrac{\deltwosqby{\gbfn{\beta}{v}{\vect{\xi}}}{\xione}
      \gsf{\pbrac{\beta}}{\pbrac{v}}\deltwosqby{\fnof{x_{j}}{\vect{\xi}}}{\xione}
      +2\deltwoby{\gbfn{\beta}{v}{\vect{\xi}}}{\xione}{\xitwo}
      \gsf{\pbrac{\beta}}{\pbrac{v}}\deltwoby{\fnof{x_{j}}{\vect{\xi}}}{
        \xione}{\xitwo}+\deltwosqby{\gbfn{\beta}{v}{\vect{\xi}}}{\xitwo}
      \gsf{\pbrac{\beta}}{\pbrac{v}}\deltwosqby{\fnof{x_{j}}{\vect{\xi}}}{\xitwo}}
  \right)\jacobian\,d\vect{\xi}
\end{multline}
and
\begin{multline}
  \delby{f^{D+1}}{\gsf{\beta}{v}}=2\dintl{\vect{0}}{\vect{1}}\left(\tau
    \pbrac{\delby{\gbfn{\beta}{v}{\vect{\xi}}}{\xione}\dotprod{\npt{\vect{x}}{
          \beta}_{,v}}{\delby{\fnof{\vect{x}}{\vect{\xi}}}{\xione}}+\delby{\gbfn{
          \beta}{v}{\vect{\xi}}}{\xitwo}\dotprod{\npt{\vect{x}}{\beta}_{,v}}{
        \delby{\fnof{\vect{x}}{\vect{\xi}}}{\xitwo}}}+\right. \\
  \left.\kappa\pbrac{\deltwosqby{\gbfn{\beta}{v}{\vect{\xi}}}{\xione}
      \dotprod{\npt{\vect{x}}{\beta}_{,v}}{\deltwosqby{\fnof{\vect{x}}{
            \vect{\xi}}}{\xione}}+2\deltwoby{\gbfn{\beta}{v}{\vect{\xi}}}{
        \xione}{\xitwo}\dotprod{\npt{\vect{x}}{\beta}_{,v}}{\deltwoby{
          \fnof{\vect{x}}{\vect{\xi}}}{\xione}{\xitwo}}+\deltwosqby{
        \gbfn{\beta}{v}{\vect{\xi}}}{\xitwo}\dotprod{\npt{\vect{x}}{\beta}_{,v}}{
        \deltwosqby{\fnof{\vect{x}}{\vect{\xi}}}{\xitwo}}}\right)\jacobian
  \,d\vect{\xi}
\end{multline}
 
Note that the above integrals are for a single element. To obtain the complete
residual gradients with respect to the optimisation parameters the
contributions from each element need to be summed.



\section{Optimisation Class}
\clearpage
